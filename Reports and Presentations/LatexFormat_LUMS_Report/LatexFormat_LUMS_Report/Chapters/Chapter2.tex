% Chapter 2

\chapter{Background} % Write in your own chapter title
\label{Chapter2}
\lhead{} % Write in your own chapter title to set the page header

\section{Literature Review}
\subsection{Application in Communication Systems}
During recent years, researchers have experimented with various neural network architectures to achieve optimal results. The simplest of architectures that have been used is a DNN with dense, fully connected layers. Generally, the number of hidden layers is kept to a minimum (3-4) as back propagation attenuates with the increasing number of layers. In communication systems, the problem is treated as a regression one instead of a classification one; hence the loss function used to train the neural networks for relevant applications is L2 norm instead of Softmax or Cross Entropy. Recent publications corroborate that for WINNER II channel model with AWGN noise in OFDM systems, the neural networks work slightly better than LS estimation or MMSE for 64 symbol pilot, and significantly better with an 8 symbol pilot without the use of a Cyclic Prefix (CP) \cite{ye2018power}.

Nonetheless, it is important to accentuate that the DNN in these works are mostly trained offline, and the bit error rates are calculated in the simulation rather than real time. More research on the use of end to end autoencoder architecture for OFDM systems has highlighted ability of the neural network to produce results comparable to MMSE and LS without the use of explicit equalizer on a single tap channel \cite{felix2018ofdm}. Moreover, the use of explicit equalizer has been observed to further improve the performance of the autoencoder \cite{felix2018ofdm}. There is also no noticeable degradation in the performance of the autoencoder if the pilot is omitted. The autoencoder architecture is also found to be robust to Carrier Frequency Offset (CFO) \cite{felix2018ofdm}. However, the discontinuity posed by the channel block to the backpropagation poses a challenge to any practical implementation of the end to end autoencoders.  

In \cite{ye2018power} and \cite{felix2018ofdm}, it is assumed that we have prior information about the channel. Specific models are used for training the neural network offline and then using it in real time. However, given the complexities of the channel, there are little prospects for improvements neural networks can bring to real-time workings of communication systems. Nonetheless, recent research into Generative Adversarial Net (GAN) is very encouraging; GAN enables the use of end to end auto encoders without any prior information about the channel \cite{ye2018channel}. 
\subsection{Application in Bioinformatics}
Parkinson’s disease is a neurodegenerative disease that can affect a person’s movement, speech, dexterity, and cognition. Physicians primarily diagnose Parkinson’s disease by performing a clinical assessment of symptoms. However, misdiagnoses are common. Research has shown that around 25\% of these diagnoses are incorrect when compared to the results of post-mortem autopsy \cite{pahwa2010early}. One factor that contributes to misdiagnoses is that the symptoms of Parkinson’s disease may not be prominent at the time the clinical assessment is performed \cite{schwab2018phonemd}. Another problem is the cumbersome process that discourages people from getting a clinical Parkinson's diagnosis, which may lead to worsening of Parkinson's until the point of no recovery. However, if we are able to achieve a good enough performance on our project, we may be able to provide a rough heuristic (if not a complete diagnosis) for people to get themselves professionally checked for PD and that kind of early diagnosis can dramatically improve people's quality of life \cite{pahwa2010early}. 

We are working on a deep learning approach to distinguish healthy patients from Parkinson’s patients using open-source data from mPower study \cite{bot2016mpower}. Machine Learning algorithms have already been applied to diagnose other diseases as well for example, Breast Cancer \cite{zheng2014breast}, cardiac risk factors \cite{oresko2010wearable}, skin cancer \cite{esteva2017dermatologist} and depression \cite{suhara2017deepmood}.

The data we are using consists of four different activities which are walking, tapping, memory and voice \cite{bot2016mpower}. Previous work on this data has achieved very impressive performance, i.e., 0.85 area under characteristic curve (AUC) \cite{schwab2018phonemd} using all modes of data and only 0.56 AUC using only voice data. This previous work uses expert hand-crafted features \cite{arora2015detecting} which may be limiting the full potential of this data as these features can be suboptimal. Our goal is to implement an end-to-end deep learning algorithm to explore the options for better discrimination between healthy and Parkinson’s patients.